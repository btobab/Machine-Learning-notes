\documentclass{report}
\usepackage{amsmath}
\usepackage{amssymb}
\begin{document}
\chapter{Expectation and Variance}
\section{Abstract}
In this issue, we mainly study some properties of Gaussian distribution
\section{Assumption}
Now given a bunch of data:
$$
X=(x_1, x_2, ..., x_N)^T
$$
$$
x_i \in \mathcal{R}^p
$$
First, we assume our model: the Gauss linear model.\\\\
To simplify the derivation of formula, we set $p$ equals $1$, so
$$
x \backsim N(\mu, \sigma^2)
$$
$$
\theta=(\mu, \sigma)
$$ 
Next, we use maximum likelihood estimation ($MLE $) to get the expectation and variance based on this bunch of data\\\\
The likelihood function is given below:
$$
\begin{aligned}
p(X|\theta)&=log(\prod_{i=1}^N \frac{1}{\sqrt{2\pi}\sigma} \exp(-\frac{(x_i-\mu)^2}{2\sigma^2})) \\
&=\sum_{i=1}^N log(\frac{1}{\sqrt{2\pi}\sigma} \exp(-\frac{(x_i-\mu)^2}{2\sigma^2}))\\
&=\sum_{i=1}^N log(\frac{1}{\sqrt{2\pi}}) - log(\sigma) - \frac{(x_i-\mu)^2}{2\sigma^2}
\end{aligned}
$$
\section{Expectation}
Next, we first use the maximum likelihood estimation to obtain the estimated value of the expected $\mu $

$$
\begin{aligned}
\mu_{MLE}
&=argmax(p(X|\theta))\\
&=argmin(\sum_{i=1}^N (x_i-\mu)^2)
\end{aligned}
$$
By deriving the formula:
$$
\begin{aligned}
\sum_{i=1}^N 2(x_i-\mu)&=0\\
\sum_{i=1}^N x_i - N \mu&=0\\
\mu_{MLE}&=\frac{1}{N} \sum_{i=1}^N x_i\\
\end{aligned}
$$

\section{Variance}
Similarly, we use maximum likelihood estimation to estimate the variance 
$\sigma $

$$
\begin{aligned}
\sigma_{MLE}
&=argmax(p(X|\theta))\\
&=argmin(\sum_{i=1}^N log(\sigma)+\frac{(x_i-\mu)^2}{2\sigma^2})
\end{aligned}
$$
Similarly, we derive the formula:
$$
\sum_{i=1}^N(\frac{1}{\sigma}-\frac{(x_i-\mu)^2}{\sigma^3})=0
$$
Finally, we get the estimated value:
$$
\sigma_{MLE}^2 = \Sigma_{MLE} = \frac{1}{N} \sum_{i=1}^N (x_i-\mu)^2
$$
\section{Bias Estimation}
To verify whether an estimate is biased or unbiased, we only need to calculate the expectation of the estimate.
\subsection{$\mu$}
$$
\begin{aligned}
E[\mu_{MLE}]
&=E[\frac{1}{N}\sum_{i=1}^N x_i]\\
&=\frac{1}{N}\sum_{i=1}^N E[x_i]\\
&=\mu
\end{aligned}
$$
So $\mu_{MLE}$ is an unbiased estimation
\subsection{$\sigma$}
First we deform the estimate of $\sigma$
$$
\begin{aligned}
\sigma_{MLE}^2
&=\frac{1}{N} \sum_{i=1} ^N (x_i - \mu_{MLE})^2\\
&=\frac{1}{N} \sum_{i=1} ^N (x_i^2 - 2x_i\mu_{MLE} + \mu_{MLE}^2)\\
&=\frac{1}{N} \sum_{i=1}^N x_i^2 - 2(\frac{1}{N} \sum_{i=1} ^N x_i) \mu_{MLE} + \mu_{MLE}^2\\
&=\frac{1}{N} \sum_{i=1}^N x_i^2 - 2\mu_{MLE}^2 + \mu_{MLE}^2\\
&=\frac{1}{N} \sum_{i=1}^N x_i^2 - \mu_{MLE}^2\\
&=(\frac{1}{N} \sum_{i=1}^N x_i^2-\mu^2) - (\mu_{MLE}^2-\mu^2)\\
\end{aligned}
$$
set $f_1=(\frac{1}{N} \sum_{i=1}^N x_i^2-\mu^2)$ , $f_2=(\mu_{MLE}^2-\mu^2)$\\\\
so:
$$
\begin{aligned}
E[f_1]
&=E[\frac{1}{N} \sum_{i=1}^N x_i^2 - \mu^2]\\
&=E[\frac{1}{N} \sum_{i=1}^N (x_i^2 - \mu^2)]\\
&=\frac{1}{N} \sum_{i=1}^N E[x_i^2] - E[\mu^2]\\
&=\frac{1}{N} \sum_{i=1}^N E[x_i^2] - \mu^2\\
&=\frac{1}{N} \sum_{i=1}^N E[x_i^2] - (E[x_i])^2\\
&=\sigma^2
\end{aligned}
$$
similarly:
$$
\begin{aligned}
E[f_2]
&=E[\mu_{MLE}^2 - \mu^2]\\
&=E[\mu_{MLE}^2 - (E[\mu_{MLE}])^2]\\
&=Var[\mu_{MLE}]\\
&=Var[\frac{1}{N} \sum_{i=1} ^N x_i]\\
&=\frac{1}{N^2} \sum_{i=1} ^N Var[x_i]\\
&=\frac{1}{N} \sigma^2
\end{aligned}
$$
finally, adding $f_1$ and $f_2$ , we get:
$$
E[\sigma_{MLE}^2]=\frac{N-1}{N} \sigma^2
$$
So our estimate of $\sigma$ from the maximum likelihood estimate is slightly smaller than the true value, so it is biased.\\\\
The unbiased estimate of $\sigma^2$ is $\frac{1}{N-1}\sum_{i=1}^N (x_i-\mu_{MLE})^2$
\end{document}