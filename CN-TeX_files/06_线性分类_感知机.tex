\documentclass{report}
\usepackage{amsmath}
\usepackage{amssymb}
\usepackage{xeCJK}
\usepackage{listings}
\setCJKmainfont{STKaiti}
\begin{document}
\chapter{感知机}
\section{摘要}
众所周知,线性分类分为两种:
\begin{itemize}
	\item 硬输出(直接输出样本的类别):
	\begin{itemize}
	\item 感知机
	\item 线性判别分析
	\end{itemize}
	\item 软输出(输出样本属于某类别的概率):
	\begin{itemize}
	\item 高斯判别分析
	\item 逻辑回归
	\end{itemize}
\end{itemize}
本期将介绍一种简单的线性二分类模型:感知机(Perceptron),它的要求比较松,只要能找到一个超平面将正负样本分割开就行。
\section{算法思想}
\subsection{错误驱动}
从字面上我们就可以看出,感知机模型的思路就是先随机初始化模型的参数,然后根据当前参数是否能够正确分割正负样本,通过错误来更新自己的参数。
\section{算法}
首先给出模型的目标函数:
$$
f(x)=sign(w^Tx)
$$
其中,$sign$是一个符号函数:
$$
sign(a)=
\begin{cases}
+1,\,\,a>0\\
-1,\,\,a\le0\\
\end{cases}
$$
那么根据上面提到的感知机的思想:错误驱动\\
我们很容易写出该模型的损失函数:
$$
L(w)=\sum_{i=1}^n I\{w^Tx_i  y_i<0 \}
$$
其中,$I$是指示函数,表示有哪些元素属于该集合。\\
而判断条件也很好理解。我们注意到:
\begin{itemize}
	\item 当$y_i>0$,即$y_i$为正例;
	\begin{itemize}
	\item 此时,若$w^Tx_i<0$,则说明该样本被错误分类($w^Tx_i y_i<0$)
	\item 若$w^Tx_i>0$,则说明该样本被正确分类($w^Tx_i y_i>0$)
	\end{itemize}
	\item 当$y_i<0$,即$y_i$为负例;
	\begin{itemize}
	\item 此时,若$w^Tx_i<0$,则说明该样本被正确分类($w^Tx_i y_i>0$)
	\item 若$w^Tx_i>0$,则说明该样本被错误分类($w^Tx_i y_i<0$
	\end{itemize}
\end{itemize}
我们最终发现,当$w^Tx_i  y_i<0$时,可以表示样本被模型错误分类。
好,我们现在再回头看损失函数,我们惊奇地发现,这个损失函数居然是不可导的,没法梯度下降了,这可肿么办尼。

因此我们放宽了条件,损失函数更为:
$$
L(w)=\sum_{(x_i,y_i)\in M} -w^Tx_i y_i
$$
$M$表示被错误分类的样本的集合。

在$w^Tx_i y_i$前面加个负号,就可以得到正的损失值了。这样就可以使用梯度下降算法更新参数$w$了。

至于这个损失函数的导数也很容易求嘛:

$$
\frac{dL(w)}{dw}=\sum_{(x_i,y_i)\in M} -x_i y_i
$$
\newpage
\section{实作}
\begin{lstlisting}[language={python}]
import os
os.chdir("../")
import numpy as np
from models.linear_models import Perceptron


model = Perceptron(10000, lr=1e-2)
x = np.linspace(0, 100, num=100)
w1, b1 = 0.1, 5
w2, b2 = 0.2, 10
epsilon = 2
k = 0.15
b = 8
w = np.asarray([-k, 1])
v1 = x * w1 + b1 + np.random.normal(scale=epsilon, size=x.shape)
v2 = x * w2 + b2 + np.random.normal(scale=epsilon, size=x.shape)
x1 = np.c_[x, v1]
x2 = np.c_[x, v2]
x = np.r_[x1, x2]
y = np.sign(x.dot(w) - b)
model.fit(x, y)
model.draw(x)
\end{lstlisting}
\end{document}