\documentclass{report}
\usepackage{amsmath}
\usepackage{amssymb}
\usepackage{xeCJK}
\usepackage{listings}
\setCJKmainfont{STKaiti}
\begin{document}
\chapter{主坐标分析}
\section{摘要}
在上期,我们学习了 PCA 公式的推导过程,但该公式在实际使用中较为麻烦。需要先求出协方差矩阵,再对其进行奇异值分解。因此更常用的方法是直接对中心化的数据集进行奇异值分解。\\
此外,使用主成分分析,我们最终得到的是新的坐标基,要对数据集进行降维,还需要再进行坐标的投影。因此本期将介绍一种相似但更为简便的方法:主坐标分析 (PCoA)
\section{算法}
\subsection{SVD and PCA}
在上一期中,我们推导出了协方差矩阵的简化形式:
$$
S=\frac{1}{N} X^T H X
$$
同时,我们也顺带推导出中心矩阵 $H^2以及H^T$ 都是其本身 $H$。\\
因此得到:
$$
S=\frac{1}{N} X^T H^T H X
$$
又因为我们可以对任何矩阵进行奇异值分解,因此我们有:
$$
HX = U \Sigma V^T
$$
因此,代入协方差矩阵中:
$$
S = V \Sigma U^T U \Sigma V^T
$$
我们知道:
$$
U^T U=I \quad V^T V = V V^T= I
$$
$\Sigma$ 为对角矩阵\\
因此:
$$
S=V\Sigma ^2 V^T
$$
写到这里,我们发现,只需对中心化的数据集进行奇异值分解,我们就可以得到协方差矩阵的特征值 $\Sigma$ 和特征向量 $V$。\\
我们计算 HXV 即可得到投影后的坐标。
\subsection{PCoA}
下面我们对 $S$ 的形式做一下颠倒,构造一个矩阵:
$$
T=HXX^TH^T
$$
与上述过程相似,我们得到:
$$
\begin{aligned}
T&=HXX^TH^T\\
&=U\Sigma V^T V\Sigma U^T\\
&=U\Sigma^2 U^T
\end{aligned}
$$
我们将投影后的坐标稍加推导:
$$
HXV=U\Sigma V^TV=U\Sigma
$$
因此主坐标分析可以直接求出投影坐标
\end{document}