\documentclass{report}
\usepackage{amsmath}
\usepackage{amssymb}
\usepackage{xeCJK}
\usepackage{listings}
\setCJKmainfont{STKaiti}
\begin{document}
\chapter{高斯判别分析}
\section{摘要}
本期我们学习线性分类-软输出-概率生成模型的一种算法:高斯判别分析(GDA)。
\section{算法思想}
在前一期我们学习的逻辑回归算法属于概率判别模型,判别模型与生成模型的区别是:
\begin{itemize}
	\item 判别模型是直接对概率 $p(y|x)$ 进行建模,求出其真实的概率值
	\item 生成模型是则是对 $p(y|x)$ 使用贝叶斯定理,转化为 $\frac{p(x|y)p(y)}{p(x)}$,因为 $p(x)$ 与 $y$ 无关,因此可以忽略,最终得到:
\end{itemize}
$$
p(y|x)\propto p(x|y)p(y)=p(x;y)
$$
因此我们关注的是 $(x,y)$ 这个联合分布,最后预测时只需比较 $p(y=0|x),p(y=1|x)$ 哪个大即可。
\section{算法}
首先,我们对模型做出一些假设:
$$
y\in \{0,1\}\quad y\sim Bernuolli(\phi)\quad p(y)=\phi^y(1-\phi)^{1-y}\\\left \{\begin{aligned}x|y=1 \ \sim \ N(\mu_1,\Sigma)\\x|y=0 \ \sim \ N(\mu_2,\Sigma)\end{aligned}\right.
$$
$$
\Longrightarrow p(x|y)=N(\mu_1,\Sigma)^yN(\mu_2,\Sigma)^{1-y}
$$
因此模型的所有参数 $\theta$ 为:
$$
\theta=(\phi, \mu_1, \mu_2, \Sigma)
$$
现在给出模型的损失函数:
$$
\begin{aligned}
J(\theta)=log(p(Y|X))&=log(\prod_{i=1}^n p(y_i|x_i))\\
&=\sum_{i=1}^n log(p(y_i|x_i))\\
\end{aligned}
$$
因此:
$$
\begin{aligned}
\hat{\theta}=argmax(J(\theta))&=argmax(\sum_{i=1}^nlog(\frac{p(x_i|y_i)p(y_i)}{p(x_i)}))\\
&=argmax(\sum_{i=1}^n log(p(x_i|y_i)p(y_i)))\\
&=argmax(\sum_{i=1}^n y_i\ log(N(\mu_1,\Sigma))+(1-y_i)\ log(N(\mu_2,\Sigma))+log(\phi^{y_i} (1-\phi)^{1-y_i}))
\end{aligned}
$$
\section{$\phi$ 的求解}
对 $\phi$ 求偏导:
$$
\sum_{i=1}^{N} \frac{y_{i}}{\phi}+\frac{y_{i}-1}{1-\phi}=0
\Longrightarrow \phi=\frac{\sum_{i=1}^{N} y_{i}}{N}=\frac{N_{1}}{N}
$$
其中,$N,N_1,N_2$ 分别为总样本的个数,正例与反例的个数
\section{$\mu$ 的求解}
然后对 $\mu_1$ 进行求解:
$$
\begin{aligned} \hat{\mu_{1}} 
&=\underset{\mu_{1}}{argmax} \sum_{i=1}^{N} y_{i} \log N\left(\mu_{1}, \Sigma\right) \\
&=\underset{\mu_1}{argmax} \sum_{i=1}^{N} y_i \log (\frac{1}{(2\pi)^{\frac{p}{2}}|\Sigma|^{\frac{1}{2}}}exp(-\frac{1}{2}(x_i-\mu_1)^T(\Sigma)^{-1}(x_i-\mu_1)))\\
&=\underset{\mu_{1}}{argmin} \sum_{i=1}^{N} y_{i}\left(x_{i}-\mu_{1}\right)^{T} \Sigma^{-1}\left(x_{i}-\mu_{1}\right)
\end{aligned}
$$
上述推导中用到了多元高斯分布的概率密度函数:
$$
p(x)=\frac{1}{(2\pi)^{\frac{p}{2}}|\Sigma|^{\frac{1}{2}}}exp(-\frac{1}{2}(x_i-\mu_1)^T(\Sigma)^{-1}(x_i-\mu_1))
$$
其中,$p$ 为随机变量的个数,读者可以根据一元高斯分布的概率密度函数进行连乘,并辅以线代的知识,就可以推出多元的公式。\\
下面对式子进行微分:
$$
\frac{\partial \Delta}{\partial \mu_1}=\sum_{i=1}^N -2y_i (\Sigma)^{-1}(x_i-\mu_1)=0\\
\Longrightarrow \mu_{1}=\frac{\sum_{i=1}^{N} y_{i} x_{i}}{\sum_{i=1}^{N} y_{i}}=\frac{\sum_{i=1}^{N} y_{i} x_{i}}{N_{1}}
$$
而由于正例与反例是对称的,因此:
$$
\mu_{2}=\frac{\sum_{i=1}^{N}\left(1-y_{i}\right) x_{i}}{N_{2}}
$$
\section{$\Sigma$ 的求解}
我们观察式子的前两项:
$$
\hat{\theta}=argmax(\sum_{i=1}^n y_i\ log(N(\mu_1,\Sigma))+(1-y_i)\ log(N(\mu_2,\Sigma))+log(\phi^{y_i} (1-\phi)^{1-y_i}))
$$
发现,当$y=0$时,第一项都为$0$;当$y=1$时,第二项都为$0$。\\
因此式子可以更为:
$$
\begin{aligned}
\hat{\theta}
&=argmax(\sum_{(x_i,y_i)\in C_1} \ log(N(\mu_1,\Sigma))+\sum_{(x_i,y_i)\in C_2}\ log(N(\mu_2,\Sigma)))\\
&=argmax(\sum_{(x_i,y_i)\in C_1} -\frac{1}{2}\log|\Sigma|-\frac{1}{2}(x_i-\mu_1)^T(\Sigma)^{-1}(x_i-\mu_1) \\
&+\sum_{(x_i,y_i)\in C_2} -\frac{1}{2}|\Sigma|-\frac{1}{2}(x_i-\mu_2)^T(\Sigma)^{-1}(x_i-\mu_2))
\end{aligned}
$$
我们观察 $(x_i-\mu)^T(\Sigma)^{-1}(x_i-\mu)$ 的形状:$(1,p)* (p,p) * (p,1)=(1,1)$,因此可以对它加上迹(tr)的符号,将其看作一个矩阵,而在迹的内部,矩阵的顺序是可以随意交换的:
$$
\begin{aligned}
\hat{\theta}
&=argmax(-\frac{N}{2}\log|\Sigma|-\frac{1}{2}tr(\sum_{(x_i,y_i)\in C_1}(x_i-\mu_1)^T(\Sigma)^{-1}(x_i-\mu_1))\\
&-\frac{1}{2}tr(\sum_{(x_i,y_i)\in C_2}(x_i-\mu_2)^T(\Sigma)^{-1}(x_i-\mu_2)))\\
&=argmax(-\frac{N}{2}\log|\Sigma|-\frac{1}{2}tr(\sum_{(x_i,y_i)\in C_1}(x_i-\mu_1)^T(x_i-\mu_1)(\Sigma)^{-1})\\
&-\frac{1}{2}tr(\sum_{(x_i,y_i)\in C_2}(x_i-\mu_2)^T(x_i-\mu_2)(\Sigma)^{-1}))\\
&=argmax(-\frac{N}{2}\log|\Sigma|-\frac{1}{2}tr(N_1 S_1(\Sigma)^{-1})
-\frac{1}{2}tr(N_2 S_2(\Sigma)^{-1}))\\
\end{aligned}
$$
其中,$S$为协方差矩阵。\\
下面对式子求偏导:
$$
\frac{\partial \Delta}{\partial \Sigma}=-\frac{1}{2}(N \frac{1}{|\Sigma|}|\Sigma|(\Sigma)^{-1}-N_1S_1(\Sigma)^{-2}-N_2S_2(\Sigma)^{-2})=0
$$
因此求解出 $\hat{\Sigma}$:
$$
N \Sigma^{-1}-N_{1} S_{1}^{T} \Sigma^{-2}-N_{2} S_{2}^{T} \Sigma^{-2}=0\\
\Longrightarrow \hat{\Sigma}=\frac{N_{1} S_{1}+N_{2} S_{2}}{N}
$$
最后,当我们要预测的时候,只需比较 $p(x|y=0)p(y=0)$ 与 $p(x|y=1)p(y=1)$哪一个更大即可。
\section{实作}
\begin{lstlisting}[language={python}]
import numpy as np
import os
os.chdir("../")
from models.linear_models import GDA

n1 = 1000
n_test = 100
x = np.linspace(0, 10, n1 + n_test)
w1, w2 = 0.3, 0.5
b1, b2 = 0.1, 0.2
x1 = x[:n1]
x_test = x[n1:]
v1 = x1 * w1 + b1
v2 = x1 * w2 + b2
cla_1 = np.c_[x1, v1]
cla_2 = np.c_[x1, v2]
l1 = np.ones(shape=(cla_1.shape[0], 1))
l2 = np.zeros(shape=(cla_2.shape[0], 1))
train_data = np.r_[cla_1, cla_2]
train_label = np.r_[l1, l2]

v_test = x_test * w2 + b2
data_test = np.c_[x_test, v_test]

model = GDA()
model.fit(train_data, train_label)
print(model.get_params())
print("accuary:", model.evaluate(data_test, 0))
\end{lstlisting}
\end{document}