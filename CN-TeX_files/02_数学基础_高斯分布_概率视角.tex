\documentclass{report}
\usepackage{amsmath}
\usepackage{amssymb}
\usepackage{xeCJK}
\setCJKmainfont{STKaiti}
\begin{document}
\chapter{概率视角}
\section{摘要}
本期我们将从概率视角观察多元高斯分布。
\section{先验知识}
$$
x \backsim N(\mu, \sigma^2)
$$
$$
\mu \in R^p, \sigma \in R^p
$$
$$
x_i \backsim N(\mu_i, \sigma_i)
$$
$$
p(x_i) = \frac{1}{\sqrt{2\pi}\sigma_i} \exp(-\frac{(x_i - \mu_i)^2}{2\sigma_i^2})
$$
\section{推导}
首先我们假设每个 $x_i$ 之间是 $iid (independent\ identically\ distribution)$ 独立同分布的。\\
即:
$$
\begin{aligned}
p(x)
&=\prod_{i=1}^p p(x_i)\\
&=\frac{1}{(2\pi)^{\frac{p}{2}}\prod_{i=1}^p \sigma_i} \exp(-\frac{1}{2}\sum_{i=1}^p (\frac{(x_i-\mu_i)^2}{\sigma_i^2}))\\
&=\frac{1}{(2\pi)^{\frac{p}{2}}|\Sigma|^{\frac{1}{2}}} \exp[-\frac{1}{2}
\left (
\begin{matrix}
x_1-\mu_1 & x_2-\mu_2 & ... & x_p - \mu_p
\end{matrix}
\right )
\left (
\begin{matrix}
\frac{1}{\sigma_1^2} & 0 & ... & 0 \\
... & ... & ... & ... \\
0 & ... & 0 & \frac{1}{\sigma_p^2}
\end{matrix}
\right )
\left (
\begin{matrix}
x_1-\mu_1\\
.&\\
.&\\
x_p-\mu_p\\
\end{matrix}
\right )]\\
&=\frac{1}{(2\pi)^{\frac{p}{2}}|\Sigma|^{\frac{1}{2}}} \exp[-\frac{1}{2}(x-\mu)^T \Sigma^{-1}(x-\mu)]
\end{aligned}
$$
以上为多元高斯分布的概率密度函数。\\\\
而我们知道 $\Sigma$ 为半正定矩阵,因此可以进行奇艺值分解。所以我们有:
$$
\begin{aligned}
\Sigma
&=UVU^T\\
&=
\left ( \begin{matrix}
u_1 & ... & u_p
\end{matrix} \right )
\left ( \begin{matrix}
\lambda_1 & 0 & ... & 0\\
... & 0 & ... & ...\\
0 & ... & ... & \lambda_p
\end{matrix} \right )
\left ( \begin{matrix}
u_1^T \\
.\\
.\\
u_p^T
\end{matrix} \right )\\
&=
\left ( \begin{matrix}
u_1\lambda_1 & ... & u_p \lambda_p\\
\end{matrix} \right )
\left ( \begin{matrix}
u_1^T\\
.\\
.\\
u_p^T
\end{matrix} \right )\\
&=\sum_{i=1}^p u_i\lambda_i u_i^T
\end{aligned}
$$
因此
$$
\begin{aligned}
\Sigma^{-1}
&=(UVU^T)^{-1}\\
&=(U^T)^{-1}V^{-1}U^{-1}\\
&=UV^{-1}U^T\\
&=\sum_{i=1}^p u_i \frac{1}{\lambda_i} u_i^T
\end{aligned}
$$
下面我们令 $\Delta = (x-\mu)^T \Sigma^{-1} (x-\mu)$\\
将上面推导的结果代入:
$$
\begin{aligned}
\Delta
&=(x-\mu)^T \Sigma^{-1} (x-\mu)\\
&=(x-\mu)^T \sum_{i=1}^p u_i \frac{1}{\lambda_i}u_i^T (x-\mu)\\
&=\sum_{i=1}^p(x-\mu)^T u_i \frac{1}{\lambda_i}u_i^T (x-\mu)
\end{aligned}
$$
下面我们令 $y_i=(x-\mu)^T u_i$\\
这里 $y_i$ 代表 $x$ 经过中心化后投影到新的正交基 $u_i$ 的坐标值。\\
所以:
$$
\Delta=\sum_{i=1}^p \frac{y_i^2}{\lambda_i}
$$
下面我们再看多元高斯分布的概率密度函数:
$$
p(x)=\frac{1}{(2\pi)^{\frac{p}{2}}|\Sigma|^{\frac{1}{2}}} \exp[-\frac{1}{2}(x-\mu)^T \Sigma^{-1}(x-\mu)]
$$
可以看到式子里与变量 $x$ 相关的只有指数。前面的系数是为了使概率和为 $1$。\\
因此高斯分布的概率与 $\Delta$ 的值直接相关。\\
我们假设 $p=2$ ,即:
$$
\frac{y_1^2}{\lambda_1}+\frac{y_2^2}{\lambda_2}=\Delta
$$
我们惊喜地发现,这与椭圆方程很像。而 $\Delta$ 的值是不固定的,因此对于不同的 $x$ ,这些样本点于平面内形成了一个个同心的椭圆。而这就是高斯分布的性质之一。
\end{document}