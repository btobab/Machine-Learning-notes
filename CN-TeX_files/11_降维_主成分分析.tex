\documentclass{report}
\usepackage{amsmath}
\usepackage{amssymb}
\usepackage{xeCJK}
\usepackage{listings}
\setCJKmainfont{STKaiti}
\begin{document}
\chapter{主成分分析}
\section{摘要}
本期我们开始学习降维的算法。\\
我们知道,解决过拟合的问题除了增加数据和正则化之外,降维是最好的方法。\\
实际上,早先前辈们就遇见过维度灾难。我们知道 $n$ 维球体的体积为$CR^n$\\
因此球体的体积与 $n$ 维超立方体的比值为
$$
\lim_{n \to +\infty}=\frac{CR^n}{2^n R^n}=0
$$
由公式我们可以看出,在高维数据中,样本的分布是相当稀疏的,超立方体的内部基本上是空心的,因此对数据的建模增大了难度。这就是所谓的维度灾难。\\
降维的方法分为:
\begin{itemize}
	\item 直接降维,特征选择
	\item 线性降维,PCA,MDS等
	\item 分线性,流形包括 Isomap,LLE 等
\end{itemize}
\section{算法思想}
对于PCA的核心思想,老师总结了一句顺口溜:一个中心,两个基本点
\begin{itemize}
	\item 一个中心:
	\begin{itemize}
	\item 将原本可能线性相关的各个特征,通过正交变换,变换为一组线性无关的特征
	\item 即对原始特征空间的重构。
	\end{itemize}
	\item 两个基本点:
	\begin{itemize}
	\item 最大投影方差
	\begin{itemize}
	\item 使数据在重构后的特征空间中更加分散(因为原始的数据都是聚为一堆分散在角落的)
	\end{itemize}
	\item 最小重构距离
	\begin{itemize}
	\item 使得数据在重构之后,损失的信息最少(即在补空间的分量更少)
	\end{itemize}
	\end{itemize}
\end{itemize}
\section{算法}
下面我们主要讲述第一个基本点:最大投影方差,其实两个基本点都是一个意思,只不过是从不同的角度对一个中心进行诠释。\\
首先是投影,关于投影的知识,我们前面已经讲过了,这里也是一样。我们假设样本点$x_i$,一个基向量$u_i$,假设$u_i^Tu_i=1$,因此可以得到样本在$u_i$这个维度的投影为
$$
project_i=x_i^Tu_i
$$
而样本经正交变换后原本有$p$个特征维度,因我们需对其降维,因此只取其前$q$个特征,而这$q$个特征都是线性无关的,因此可以将这些投影直接相加,得到样本在新的特征空间的投影。\\
注意在求投影之前先将数据做中心化,因此数据的均值归零,求投影的方差可以直接平方。\\
综上,我们得到了目标函数:
$$
J=\frac{1}{N} \sum_{i=1}^{N} \sum_{j=1}^{q}\left(\left(x_{i}-\bar{x}\right)^{T} u_{j}\right)^{2}
$$
下面对目标函数稍作推导:\\
因为$((x_i-\bar{x})^Tu_j)$的形状为$(1,p)* (p,1)=(1,1)$,因此可以对其做转置:
$$
\begin{aligned}
J&=\frac{1}{N} \sum_{i=1}^{N} \sum_{j=1}^{q}\left(\left(x_{i}-\bar{x}\right)^{T} u_{j}\right)^{2}\\
&=\frac{1}{N} \sum_{i=1}^{N} \sum_{j=1}^{q}(u_{j}^T(x_{i}-\bar{x}))^{2}\\
&=\frac{1}{N} \sum_{i=1}^{N} \sum_{j=1}^{q}u_{j}^T(x_{i}-\bar{x}))(x_{i}-\bar{x})^T u_j\\
&=\sum_{j=1}^{q} u_{j}^T(\frac{1}{N} \sum_{i=1}^{N} (x_{i}-\bar{x}))(x_{i}-\bar{x})^T) u_j\\
&=\sum_{j=1}^q u_j^T S u_j\\
\end{aligned}
$$
别忘了我们还有一个限制条件:$s.t\ u_j^T u_j=1$\\
因此可以使用拉格朗日乘子法:
$$
\underset{u_{j}}{\operatorname{argmax}} L\left(u_{j}, \lambda\right)=\underset{u_{j}}{\operatorname{argmax}} u_{j}^{T} S u_{j}+\lambda\left(1-u_{j}^{T} u_{j}\right)
$$
对上式求导:
$$
\frac{\partial \Delta}{\partial u_j}=2S u_j -2\lambda u_j=0
$$
得到结果:
$$
S u_j = \lambda u_j
$$
可以看出,变换后的基向量实际上为协方差矩阵的特征向量,$\lambda$ 为$S$的特征值
实际上,对于协方差矩阵的求解也可以化简:
$$
\begin{aligned}
S &=\frac{1}{N} \sum_{i=1}^{N}\left(x_{i}-\bar{x}\right)\left(x_{i}-\bar{x}\right)^{T} \\
&=\frac{1}{N}\left(x_{1}-\bar{x}, x_{2}-\bar{x}, \cdots, x_{N}-\bar{x}\right)\left(x_{1}-\bar{x}, x_{2}-\bar{x}, \cdots, x_{N}-\bar{x}\right)^{T} \\
&=\frac{1}{N}\left(X^{T}-\frac{1}{N} X^{T} I_{N} I_{N}^{T}\right)\left(X^{T}-\frac{1}{N} X^{T} I_{N} I_{N}^{T}\right)^{T} \\
&=\frac{1}{N} X^{T}\left(E_{N}-\frac{1}{N} I_{N} I_{N}^T\right)\left(E_{N}-\frac{1}{N} I_{N} I_{N}^T\right)^{T} X \\
&=\frac{1}{N} X^{T} H_{N} H_{N}^{T} X \\
&=\frac{1}{N} X^{T} H_{N} H_{N} X=\frac{1}{N} X^{T} H X
\end{aligned}
$$
这里$H$是一个特殊的矩阵,被称为中心矩阵。
$$
H=E_N - \frac{1}{N}I_N I_N^T
$$
因此,在实作中,我们只需要用上式求出协方差矩阵,然后对其做正交分解得到特征值与特征向量即可。
\newpage
\section{实作}
\begin{lstlisting}[language={python}]
import numpy as np
import os
os.chdir("../")
from models.decompose_models import PCA

k, b = 3, 4
x = np.linspace(0, 10, 100)
y = x * k + b
x += np.random.normal(scale=0.3, size=x.shape)
data = np.c_[x, y]

model = PCA()
model.fit(data)
model.draw(data)
\end{lstlisting}
\end{document}