\documentclass{report}
\usepackage{amsmath}
\usepackage{amssymb}
\usepackage{xeCJK}
\setCJKmainfont{STKaiti}
\begin{document}
\chapter{边缘概率和条件概率}
\section{摘要}
本节我们学习多元高斯分布的边缘概率和条件概率
\section{先验知识}
在上一节中,我们推导了多元高斯分布的概率密度函数:
$$
x \backsim N(\mu, \Sigma) = \frac{1}{(2\pi)^{\frac{p}{2}}|\Sigma|^\frac{1}{2}}\exp(-\frac{1}{2}(x-\mu)^T \Sigma^{-1}(x-\mu))
$$
现在我们将随机变量 $x$ 拆分为两部分:
$$
x \in R^p \quad x_a \in R^m \quad x_b \in R^n \quad m+n=p \quad 
$$
$$
x = 
\left (
\begin{matrix}
x_a \\
x_b \\
\end{matrix}
\right )
\quad 
\mu = 
\left (
\begin{matrix}
\mu_a \\
\mu_b \\
\end{matrix}
\right ) \quad
\Sigma=
\left (
\begin{matrix}
\Sigma_{aa} & \Sigma_{ab}\\
\Sigma_{ba} & \Sigma_{bb}\\
\end{matrix}
\right )
$$
\section{定理}
$$
X \backsim N(\mu, \Sigma) \quad Y=AX+B  \Longrightarrow Y \backsim N(A\mu+B, A \Sigma A^T)
$$
\section{推导边缘概率}
$$
x_a = \left (\begin{matrix}I & 0\end{matrix}\right )\left (\begin{matrix}x_a \\x_b\\\end{matrix}\right ) + 0
$$
$$
E[x_a] = \left (\begin{matrix}I & 0\end{matrix}\right )\left (\begin{matrix}\mu_a \\\mu_b \\\end{matrix}\right ) = \mu_a
$$
$$
\begin{aligned}Var[x_a] &= \left (\begin{matrix}I & 0\end{matrix}\right )\left (\begin{matrix}\Sigma_{aa} & \Sigma_{ab} \\\Sigma_{ba} & \Sigma_{bb} \\\end{matrix}\right )\left (\begin{matrix}I \\0\end{matrix}\right ) \\&= \left (\begin{matrix}\Sigma_{aa} & \Sigma_{ab} \end{matrix}\right )\left (\begin{matrix}I \\0\end{matrix}\right )\\&=\Sigma_{aa}\end{aligned}
$$
$$
\therefore x_a \backsim N(\mu_a, \Sigma_{aa})
$$
\section{推导条件概率}
我们设:
$$
\begin{cases}
x_{b.a}=x_b - \Sigma_{ba} \Sigma_{aa}^{-1} x_a \\
\mu_{b.a} = \mu_b - \Sigma_{b.a} \Sigma_{aa}^{-1} \mu_a \\
\Sigma_{bb.a}=\Sigma_{bb} - \Sigma_{ba} \Sigma_{aa}^{-1} \Sigma_{ab}
\end{cases}
$$
$$
\begin{aligned}
x_{b.a} &= x_b - \Sigma_{ba} \Sigma_{bb}^{-1} x_a \\
&=
\left (
\begin{matrix}
- \Sigma_{ba} \Sigma_{bb}^{-1} & I\\
\end{matrix}
\right )
\left (
\begin{matrix}
x_a \\
x_b \\
\end{matrix}
\right ) + 0
\end{aligned}
$$
$$
\begin{aligned}
E[x_{b.a}]
&=
\left (
\begin{matrix}
-\Sigma_{ba} \Sigma_{aa}^{-1} & I
\end{matrix}
\right )
\left (
\begin{matrix}
\mu_a \\
\mu_b \\
\end{matrix}
\right )\\
&=\mu_b - \Sigma_{ba} \Sigma_{aa}^{-1} \mu_a\\
&=\mu_{b.a}
\end{aligned}
$$
$$
\begin{aligned}
Var[x_{b.a}]
&= \left ( \begin{matrix}
-\Sigma_{ba} \Sigma_{aa}^{-1} & I
\end{matrix} \right )
\left ( \begin{matrix}
\Sigma_{aa} & \Sigma_{ab} \\
\Sigma_{ba} & \Sigma_{bb} \\
\end{matrix} \right )
\left ( \begin{matrix}
-\Sigma_{aa}^{-1} \Sigma_{ba}^T \\
I
\end{matrix} \right ) \\
&=\left ( \begin{matrix}
0 & \Sigma_{bb} - \Sigma_{ba} \Sigma_{aa}^{-1} \Sigma_{ab} \\
\end{matrix} \right ) 
\left ( \begin{matrix}
-\Sigma_{aa}^{-1} \Sigma_{ba}^T \\backsim
I
\end{matrix} \right ) \\
&=\Sigma_{bb} - \Sigma_{ba} \Sigma_{aa}^{-1} \Sigma_{ab}\\
&=\Sigma_{bb.a}
\end{aligned}
$$
$$
\therefore x_{b.a} \backsim N(\mu_{b.a}, \Sigma_{bb.a})
$$
$$
\begin{cases}
\mu_{b.a} = \mu_b - \Sigma_{ba} \Sigma_{aa}^{-1} \mu_a \\
\Sigma_{bb.a}=\Sigma_{bb} - \Sigma_{ba} \Sigma_{aa}^{-1} \Sigma_{ab}
\end{cases}
$$
我们将式子稍作变形:
$$
\begin{aligned}
x_{b.a}
&=x_b - \Sigma_{ba} \Sigma_{aa}^{-1} x_a\\
x_b | x_a&=x_{b.a} + \Sigma_{ba} \Sigma_{aa}^{-1} x_a\\
&=I x_{b.a} + C\\
\end{aligned}
$$
此处,协方差矩阵的部分 $\Sigma_{ba} \ \Sigma_{aa}$ 都是可以通过计算得到的,因此可以视为常数。\\
而我们此处要求的是 $x_b | x_a$ ,因此 $x_a$ 也是已知的,因此上式的第二项可以视为常数。\\\\
因此:
$$
E[x_b|x_a]=I E[x_{b.a}] + C = \mu_{b.a} + \Sigma_{ba} \Sigma_{aa}^{-1} x_a
$$
$$
Var[x_b|x_a]=IVar[x_{b.a}]I^T=\Sigma_{bb.a}
$$
因此我们得出了高斯分布的条件概率:
$$
x_b|x_a \backsim N(\mu_{b.a} + \Sigma_{ba} \Sigma_{aa}^{-1} x_a, \Sigma_{bb.a})
$$
\end{document}