\documentclass{report}
\usepackage{amsmath}
\usepackage{amssymb}
\usepackage{xeCJK}
\usepackage{listings}
\setCJKmainfont{STKaiti}
\begin{document}
\section{逻辑回归}
\subsection{摘要}
本期我们将学习二分类-软输出的一种算法:逻辑回归。该算法主要是依托于一个激活函数:sigmoid,因为这个函数的值域为$(0, 1)$,因此可以近似表示概率值。
\subsection{本质}
以下为 shuhuai 老师的讲义上的解释:
\begin{quotation}
	有时候我们只要得到一个类别的概率,那么我们需要一种能输出$(0, 1)$区间的值的函数。考虑两分类模型,我们利用判别模型,希望对$p(C|x)$建模,利用贝叶斯定理:
$$
p\left(C_{1} \mid x\right)=\frac{p\left(x \mid C_{1}\right) p\left(C_{1}\right)}{p\left(x \mid C_{1}\right) p\left(C_{1}\right)+p\left(x \mid C_{2}\right) p\left(C_{2}\right)}
$$
取 $a=\ln \frac{p\left(x \mid C_{1}\right) p\left(C_{1}\right)}{p\left(x \mid C_{2}\right) p\left(C_{2}\right)}$, 于是:
$$
p\left(C_{1} \mid x\right)=\frac{1}{1+\exp (-a)}
$$
上面的式子叫Logistic\ Sigmoid 函数,其参数表示了两类联合概率比值的对数。在判别式中,不关心这个参数的具体值,模型假设直接对 $a$ 进行。
\end{quotation}
当然了,老师高端的解释看不懂也没关系,我们只需要知道,现在我们有了一个激活函数 sigmoid,它可以用来得到一个类别的概率。
\subsection{算法}
首先我们给出逻辑回归的模型假设:
$$
f(x)=\sigma(w^Tx)
$$
其中,$\sigma(a)=sigmoid(a)$,我们一般用 $\sigma$ 来表示激活函数\\
于是,通过寻找 $w$ 的最佳值,则可以确定在该模型假设下的最佳模型。\\
概率判别模型常用极大似然估计来确定参数。\\
为了确定似然函数,我们先做一些标记:
$$
p_1=\sigma(w^Tx) \quad p_0=1-p_1
$$
其中 $p_1$ 为 $x$ 属于$1$类的概率,$p_0$ 为 $x$ 属于$0$类的概率\\
下面我们就可以给出该模型的似然函数了:
$$
p(y|w;x)=p_1^yp_0^{1-y}
$$
这个似然函数看上去操作有点骚,看不懂,其实也蛮合理的:
\begin{itemize}
	\item 当$y$为$1$时:$p(y|w;x)=p_1^1p_0^0=p_1$
	\item 当$y$为$0$时:$p(y|w;x)=p_1^0p_0^1=p_0$
\end{itemize}
好,下面我们就可以使用极大似然估计来确定参数了
\begin{equation}
\begin{aligned}
\hat{w}=argmax(J(w))&=argmax(p(Y|w;X))\\
&=argmax(log(p(Y|w;X)))\\
&=argmax(log(\prod_{i=1}^n p(y_i|w;x_i)))\\
&=argmax(\sum_{i=1}^n log(p(y_i)|w;x_i))\\
&=argmax(\sum_{i=1}^n y\ log\, p_1+(1-y)log\,p_0)
\end{aligned}
\end{equation}
注意到,这个表达式是交叉熵表达式的相反数乘 N,MLE 中的对数也保证了可以和指数函数相匹配, 从而在大的区间汇总获取稳定的梯度。\\
对上式求导,我们注意到:
$$
p_1'=p_1(1-p_1)
$$
当然这个也很容易得到,就是链式法则嘛,稍微细心一点就可以求出来了。\\
最后我们求出结果:
$$
\frac{\partial}{\partial w}J(w)=\sum_{i=1}^{N}\left(y_{i}-p_{1}\right) x_{i}
$$
最后还有一点要注意,我们是要求得$p(p|w;x)$的最大值,因此我们需要使用梯度上升,而不是梯度下降,当然两者也差不多,加个负号而已。
\subsection{实作}
\begin{lstlisting}[language={python}]
import os
os.chdir("../")
from models.linear_models import Logistic_regression
import numpy as np
import warnings
warnings.filterwarnings("ignore")


epsilon = 1
num_test = 100
num_base = 1000
ratio = 0.6
k1, k2 = 3, 5
b1, b2 = 1, 2
X = np.linspace(0, 100, num_base)
X_train = X[:-num_test]
X_test = X[-num_test:]
v1 = X_train[:round(len(X_train) * ratio)] * k1 + b1
v2 = X_train[round(len(X_train) * ratio):] * k2 + b2
v1 += np.random.normal(scale=epsilon, size=v1.shape)
v2 += np.random.normal(scale=epsilon, size=v2.shape)
value = np.r_[v1, v2]
data = np.c_[X_train, value]
l1 = np.ones_like(v1)
l2 = np.zeros_like(v2)
label = np.r_[l1, l2]
v_test_c1 = X_test * k1 + b1
l_test_c1 = np.ones_like(v_test_c1)
data_test = np.c_[X_test, v_test_c1]

model = Logistic_regression(10, 1000, lr=1e-3)
model.fit(data, label)
print(model.get_params())
print(model.predict(data_test, l_test_c1))
\end{lstlisting}
\end{document}