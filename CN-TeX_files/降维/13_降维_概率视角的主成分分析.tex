\documentclass{report}
\usepackage{amsmath}
\usepackage{amssymb}
\usepackage{hyperref}
\usepackage{xeCJK}
\setCJKmainfont{STKaiti}

\begin{document}
\section{概率视角的主成分分析}
在本节,我们将学习概率视角的PCA,这被称为P-PCA。
\subsection{已知}
$$
\begin{aligned}
x \in R^p, \ z \in R^q, \ q < p\\
\begin{cases}
z \backsim N(0_q, I_q)\\
x = wz + \mu + \epsilon\\
\epsilon \backsim N(0_p, \sigma^2 I_p)
\end{cases}
\end{aligned}
$$
\subsection{所求}
$$
x|z, \ x, \ z|x
$$
\subsection{推导}
\subsubsection{推导 $x$}
\begin{equation}
\begin{aligned}
E[x]
&= w E[z] + E[\mu] + E[\epsilon]\\
&=\mu
\end{aligned}
\end{equation}
\begin{equation}
\begin{aligned}
Var[x] &= w Var[z] w^T + Var[\mu] + Var[\epsilon]\\
&=ww^T + 0 + \sigma^2 I_p\\
&=ww^T + \sigma^2 I_p
\end{aligned}
\end{equation}
$$
\therefore x \backsim N(\mu, ww^T + \sigma^2 I_p)
$$
\subsubsection{推导 $x|z$}
\paragraph{构造分布 $t$\\}
$$
t = 
\left (
\begin{matrix}
x\\
z
\end{matrix}
\right )
$$
\begin{equation}
\begin{aligned}
E[t] &= 
\left (
\begin{matrix}
E[x]\\
E[z]
\end{matrix}
\right )\\
&=
\left (
\begin{matrix}
\mu l_p\\
0_q
\end{matrix}
\right )
\end{aligned}
\end{equation}
\begin{equation}
\begin{aligned}
cov(x,z)
&= E[(x - E[x])(z - E[z])^T]\\
&= E[(x - \mu)z^T]\\
&= E[(wz + \epsilon)z^T]\\
&= E[wzz^T] + E[\epsilon z^T]\\
&= w E[(z - E[z])(z - E[z])^T] + E[\epsilon z^T]\\
&= w Var[z] + 0\\
&= w I_q\\
&= w
\end{aligned}
\end{equation}
\begin{equation}
\begin{aligned}
\therefore 
Var[t] &= 
\left (
\begin{matrix}
ww^T + \sigma^2 I_p & w\\
w^T & I_q
\end{matrix}
\right )
\end{aligned}
\end{equation}
$$
\therefore 
t \backsim 
N(
\left [
\begin{matrix}
\mu l_p\\
0_q
\end{matrix}
\right ],
\left [
\begin{matrix}
ww^T + \sigma^2 I_p & w\\
w^T & I_q
\end{matrix}
\right ]
)
$$
\paragraph{构造分布 $x.z$\\}
我们设
$$\ x.z = x - \Sigma_{xz} \Sigma_{zz}^{-1}z$$
\begin{equation}
\begin{aligned}
\therefore x.z &= x - wz\\
&=
\left (
\begin{matrix}
I_p & -w
\end{matrix}
\right )
\left (
\begin{matrix}
x \\
z
\end{matrix}
\right )\\
&=\left (
\begin{matrix}
I_p & -w
\end{matrix}
\right )
t
\end{aligned}
\end{equation}
\begin{equation}
\begin{aligned}
\therefore 
E[x.z]&=
\left (
\begin{matrix}
I_p & -w
\end{matrix}
\right ) E[t]\\
&= 
\left (
\begin{matrix}
I_p & -w
\end{matrix}
\right )
\left (
\begin{matrix}
\mu l_p\\
0_q
\end{matrix}
\right )\\
&=\mu l_p
\end{aligned}
\end{equation}

\begin{equation}
\begin{aligned}
Var[x.z] 
&= 
\left (
\begin{matrix}
I_p & -w
\end{matrix}
\right )
Var[t]
 \left (
\begin{matrix}
I_p \\ -w^T
\end{matrix}
\right )\\
&= 
\left (
\begin{matrix}
I_p & -w
\end{matrix}
\right )
\left (
\begin{matrix}
ww^T + \sigma^2 I_p & w\\
w^T & I_q
\end{matrix}
\right )
\left (
\begin{matrix}
I_p \\ -w^T
\end{matrix}
\right )\\
&=
\left (
\begin{matrix}
\sigma^2 I_p & 0
\end{matrix}
\right )
\left (
\begin{matrix}
I_p \\ -w^T
\end{matrix}
\right )\\
&=\sigma^2 I_p
\end{aligned}
\end{equation}
\\
\begin{equation}
\therefore 
x.z \backsim N(\mu l_p, \sigma^2 I_p)
\end{equation}
\subsubsection{推导 $x|z$}
\begin{equation}
\begin{aligned}
\because 
x.z &= x - \Sigma_{xz} \Sigma_{zz}^{-1} z\\
\therefore 
x|z &= x.z + \Sigma_{xz} \Sigma_{zz}^{-1}z\\
&= x.z + wz
\end{aligned}
\end{equation}
这里 $z$ 是已知的, 因此我们可以将 $wz$ 视为常数。
\begin{equation}
\begin{aligned}
E[x|z] &= E[x.z] + wz\\
&= \mu l_p + wz
\end{aligned}
\end{equation}
\begin{equation}
\begin{aligned}
Var[x|z] &= Var[x.z] + 0\\
&= \sigma^2 I_p
\end{aligned}
\end{equation}\\
\begin{equation}
\therefore
x|z \backsim N(\mu l_p + wz, \sigma^2 I_p)
\end{equation}

\subsubsection{推导 $z|x$}
我们设
\begin{equation}
\begin{aligned}
z.x &= z - \Sigma_{zx} \Sigma_{xx}^{-1} x\\
&= z - w^T \Sigma_{xx}^{-1} x\\
\Sigma &= \Sigma_{xx} \\
\therefore 
z.x &= z - w^T \Sigma^{-1}x
\end{aligned}
\end{equation}
让我们对上式稍作变形:
\begin{equation}
\begin{aligned}
z.x &= z - w^T \Sigma^{-1}x\\
&=
\left (
\begin{matrix}
-w^T \Sigma^{-1} & I
\end{matrix}
\right )
\left (
\begin{matrix}
x\\z
\end{matrix}
\right )\\
&= \left (
\begin{matrix}
-w^T \Sigma^{-1} & I
\end{matrix}
\right )
t
\end{aligned}
\end{equation}
\begin{equation}
\begin{aligned}
E[z.x] &=
\left (
\begin{matrix}
-w^T \Sigma^{-1} & I
\end{matrix}
\right )
\left (
\begin{matrix}
\mu l_p \\ 0_q
\end{matrix}
\right )\\
&= -\mu w^T \Sigma^{-1}
\end{aligned}
\end{equation}
\begin{equation}
\begin{aligned}
Var[z.x] &=
\left (
\begin{matrix}
-w^T \Sigma^{-1} & I
\end{matrix}
\right )
\left (
\begin{matrix}
\Sigma & w\\
w^T & I
\end{matrix}
\right )
\left (
\begin{matrix}
-(\Sigma^{-1})^T w \\ I
\end{matrix}
\right )\\
&= 
\left (
\begin{matrix}
0 & -w^T \Sigma^{-1}w + I
\end{matrix}
\right )
\left (
\begin{matrix}
- \Sigma^{-1} w\\ I
\end{matrix}
\right )\\
&= -w^T \Sigma^{-1}w + I
\end{aligned}
\end{equation}\\
\begin{equation}
\therefore z.x \backsim N(-\mu w^T \Sigma^{-1}, -w^T \Sigma^{-1} w + I)
\end{equation}\\
\begin{equation}
\begin{aligned}
\because 
z.x &= z - w^T \Sigma^{-1} x\\
\therefore 
z|x &= z.x + w^T \Sigma^{-1} x\\
\end{aligned}
\end{equation}
\begin{equation}
\begin{aligned}
E[z|x] 
&= E[z.x] + w^T \Sigma^{-1} x\\
&= -\mu w^T \Sigma^{-1} + w^T \Sigma^{-1} x\\
&= w^T \Sigma^{-1} (x - \mu l_p)\\
&= w^T (ww^T + \sigma^2 I)^{-1}(x - \mu l_p)
\end{aligned}
\end{equation}
\begin{equation}
\begin{aligned}
Var[z|x] 
&= Var[z.x] + 0\\
&= -w^T \Sigma^{-1} w + I\\
&= -w^T (ww^T + \sigma^2 I)^{-1} w + I\\
\end{aligned}
\end{equation}\\
\begin{equation}
\therefore
z|x \backsim N(w^T (ww^T + \sigma^2 I)^{-1}(x - \mu l_p), -w^T (ww^T + \sigma^2 I)^{-1} w + I)
\end{equation}
\subsection{总结}
我们需要搜索的参数是:
$$
\{ w, \ \sigma , \ \mu \}
$$
我们可以用EM算法求得他们,通过最大化 $p(x|z)$,这意味着:
\begin{itemize}
	\item 训练阶段我们计算 $p(x|z)$
	\item 推理阶段我们计算 $p(z|x)$
\end{itemize}
\end{document}