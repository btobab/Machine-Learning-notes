\documentclass{report}
\usepackage{amsmath}
\usepackage{amssymb}
\usepackage{xeCJK}
\setCJKmainfont{STKaiti}
\begin{document}
\section{期望方差}
\subsection{摘要}
本期我们主要学习高斯分布的一些性质
\subsection{假设}
现在我们有一堆数据:
$$
X=(x_1, x_2, ..., x_N)^T
$$
$$
x_i \in R^p
$$
首先给出我们的模型:高斯线形模型。\\
这里我们为了简化起见,将 $p$ 设为 $1$, 因此 
$$
x \backsim N(\mu, \sigma^2)
$$
$$
\theta=(\mu, \sigma)
$$ 
接下来我们根据这堆数据,通过极大似然估计( MLE )得出其期望与方差\\
下面我们给出似然函数:
$$
\begin{aligned}
p(X|\theta)&=log(\prod_{i=1}^N \frac{1}{\sqrt{2\pi}\sigma} \exp(-\frac{(x_i-\mu)^2}{2\sigma^2})) \\
&=\sum_{i=1}^N log(\frac{1}{\sqrt{2\pi}\sigma} \exp(-\frac{(x_i-\mu)^2}{2\sigma^2}))\\
&=\sum_{i=1}^N log(\frac{1}{\sqrt{2\pi}}) - log(\sigma) - \frac{(x_i-\mu)^2}{2\sigma^2}
\end{aligned}
$$
\subsection{期望}
下面我们首先使用极大似然估计得出期望 $\mu$ 的估计值
$$
\begin{aligned}
\mu_{MLE}
&=argmax(p(X|\theta))\\
&=argmin(\sum_{i=1}^N (x_i-\mu)^2)
\end{aligned}
$$
对式子求导得到:
$$
\sum_{i=1}^N 2(x_i-\mu)=0
$$
$$
\sum_{i=1}^N x_i - N \mu=0
$$
$$
\mu_{MLE}=\frac{1}{N} \sum_{i=1}^N x_i
$$
\subsection{方差}
同样的,我们使用极大似然估计得出方差 $\sigma$ 的估计值
$$
\begin{aligned}
\sigma_{MLE}
&=argmax(p(X|\theta))\\
&=argmin(\sum_{i=1}^N log(\sigma)+\frac{(x_i-\mu)^2}{2\sigma^2})
\end{aligned}
$$
同样的,我们对式子求导得到:
$$
\sum_{i=1}^N[\frac{1}{\sigma}-\frac{(x_i-\mu)^2}{\sigma^3}]=0
$$
最后,我们得到估计值:
$$
\sigma_{MLE}^2 = \Sigma_{MLE} = \frac{1}{N} \sum_{i=1}^N (x_i-\mu)^2
$$
\subsection{偏置估计}
要验证一个估计值是有偏估计还是无偏估计,我们只需计算该估计值的期望即可。
\subsubsection{$\mu$}
$$
\begin{aligned}
E[\mu_{MLE}]
&=E[\frac{1}{N}\sum_{i=1}^N x_i]\\
&=\frac{1}{N}\sum_{i=1}^N E[x_i]\\
&=\mu
\end{aligned}
$$
因此 $\mu_{MLE}$ 为无偏估计
\subsubsection{$\sigma$}
首先我们对 $\sigma$ 的估计值进行变形
$$
\begin{aligned}
\sigma_{MLE}^2
&=\frac{1}{N} \sum_{i=1} ^N (x_i - \mu_{MLE})^2\\
&=\frac{1}{N} \sum_{i=1} ^N (x_i^2 - 2x_i\mu_{MLE} + \mu_{MLE}^2)\\
&=\frac{1}{N} \sum_{i=1}^N x_i^2 - 2(\frac{1}{N} \sum_{i=1} ^N x_i) \mu_{MLE} + \mu_{MLE}^2\\
&=\frac{1}{N} \sum_{i=1}^N x_i^2 - 2\mu_{MLE}^2 + \mu_{MLE}^2\\
&=\frac{1}{N} \sum_{i=1}^N x_i^2 - \mu_{MLE}^2\\
&=(\frac{1}{N} \sum_{i=1}^N x_i^2-\mu^2) - (\mu_{MLE}^2-\mu^2)\\
\end{aligned}
$$
令 $f_1=(\frac{1}{N} \sum_{i=1}^N x_i^2-\mu^2)$ , $f_2=(\mu_{MLE}^2-\mu^2)$\\
所以:
$$
\begin{aligned}
E[f_1]
&=E[\frac{1}{N} \sum_{i=1}^N x_i^2 - \mu^2]\\
&=E[\frac{1}{N} \sum_{i=1}^N (x_i^2 - \mu^2)]\\
&=\frac{1}{N} \sum_{i=1}^N E[x_i^2] - E[\mu^2]\\
&=\frac{1}{N} \sum_{i=1}^N E[x_i^2] - \mu^2\\
&=\frac{1}{N} \sum_{i=1}^N E[x_i^2] - (E[x_i])^2\\
&=\sigma^2
\end{aligned}
$$
类似的:
$$
\begin{aligned}
E[f_2]
&=E[\mu_{MLE}^2 - \mu^2]\\
&=E[\mu_{MLE}^2 - (E[\mu_{MLE}])^2]\\
&=Var[\mu_{MLE}]\\
&=Var[\frac{1}{N} \sum_{i=1} ^N x_i]\\
&=\frac{1}{N^2} \sum_{i=1} ^N Var[x_i]\\
&=\frac{1}{N} \sigma^2
\end{aligned}
$$
最后,将 $f_1$ 与 $f_2$ 相加,得到:
$$
E[\sigma_{MLE}^2]=\frac{N-1}{N} \sigma^2
$$
因此我们通过极大似然估计得到的 $\sigma$ 的估计值比真实值略小,所以为有偏估计
而 $\sigma^2$ 的无偏估计为 $\frac{1}{N-1}\sum_{i=1}^N (x_i-\mu_{MLE})^2$

\end{document}